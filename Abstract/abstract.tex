% ************************** Thesis Abstract *****************************
% Use `abstract' as an option in the document class to print only the titlepage and the abstract.


\begin{abstract}
Digital data play a crucial role in the information and communication technology (ICT) society: they are managed by business and governmental applications, by all kind of applications on the Web, and are 
fundamental in all relationships between governments, business, and citizens. 

Data quality can be defined as "the measure of the agreement between the data views presented by an information system and that same data in the real world". 
Data quality is a multidimensional construct consisting of different data quality dimensions such as accuracy, completeness, consistency, and currency. ~\cite{Heinrich:2018:RDQ:3155015.3148238}

Furthermore, quality of data is also a significant usage for operational process of business and organizations. Some disasters are due to the presence of data quality problems, among them the use 
of inaccurate, incomplete, out-of-data. 

As a consequence, the overall quality of the information that flows between information systems may rapidly degrade over time if both process and their inputs are not themselves subject to quality 
control. On the other hand, the same networked information system offers new opportunities for data quality management, including possibility of selecting sources with better quality data, and of comparing sources for
the purpose of error localization and correction, thus facilitating the control and improvement of data quality in the system.

Due to the described above motivations, researchers and organizations more and more need to understand and solve data quality problems, and thus answering the following questions: 
What is in essence, data quality? Which techniques, methodologies, and data quality issues are at a consolidated stage?

In this paper, we first review relevant works and discuss machine learning techniques, tools and statistical models. Second, we offer a creative data profiling framework based deep learning 
and statistical model algorithms for improving data quality.

\end{abstract}



\smallskip
\noindent \textbf{{Keywords:} {Deep Learning}, {Statistical Quality Control}, {Machine Learning}, {Data Cleaning}}







